\documentclass{beamer}

% Thème Beamer
\usetheme{Madrid}
\usecolortheme{seahorse}

% Packages utiles
\usepackage[utf8]{inputenc}
\usepackage[T1]{fontenc}
\usepackage{amsmath, amssymb, amsthm}
\usepackage{graphicx}
\usepackage{hyperref}

% Métadonnées
\title{Géométrie des Données et Apprentissage Machine}
\subtitle{Module 1 -- Introduction enrichie avec exercices}
\author{Youssef MESRI - MINES Paris - PSL}
\date{\today}

\begin{document}

% Page de titre
\begin{frame}
  \titlepage
\end{frame}

% Sommaire
\begin{frame}{Plan}
  \tableofcontents
\end{frame}


%================= MODULE 4 =================
\section{Module 4 : Topologie des données}

\begin{frame}{Module 4 : plan}
\begin{itemize}
  \item Introduction à la topologie des données
  \item Variétés riemanniennes
  \item Transport optimal
  \item Applications à l’IA générative
  \item TP : Mapper (KeplerMapper), mini-projet transport optimal (POT library).
\end{itemize}
\end{frame}
%================= Slide 26 : Topologie des données =================
\begin{frame}{Topologie des données}
\begin{itemize}
  \item Étudier la forme globale des données : trous, cycles, composantes
  \item Techniques :
  \begin{enumerate}
    \item \textbf{Mapper} : visualisation simplifiée d’un nuage de points
    \item \textbf{Persistent Homology} : détecter des features topologiques robustes
  \end{enumerate}
  \item Diagramme de persistance : barres représentant la durée de vie des composantes
\end{itemize}
\end{frame}

%================= Slide 27 : Exemple diagramme de persistance =================
\begin{frame}{Exemple : Persistent Homology}
\begin{center}
\texttt{
Points → Graph simplicial → Filtration → Barcodes
}
\begin{itemize}
  \item Chaque barre = un trou ou composante connectée
  \item Longue barre → feature robuste
  \item Courte barre → bruit
\end{itemize}
\end{center}
\end{frame}

%================= Slide 28 : Variétés riemanniennes =================
\begin{frame}{Variétés riemanniennes}
\begin{itemize}
  \item Une variété lisse $(\mathcal{M}, g)$ avec métrique $g$ définit longueur et angles
  \item Distance géodésique $d_\mathcal{M}(x,y)$ = longueur minimale d’un chemin sur $\mathcal{M}$
  \item Exemples : sphère, tore, espace de rotations SO(3)
  \item Applications : données sur sphère, pose 3D, embedding non-linéaire
\end{itemize}
\end{frame}

%================= Slide 29 : Transport optimal =================
\begin{frame}{Transport optimal}
\begin{itemize}
  \item Comparer deux distributions $\mu$ et $\nu$
  \item Distance de Wasserstein :
  \[
    W_p(\mu, \nu) = \Bigg( \inf_{\gamma \in \Pi(\mu, \nu)} \int \|x - y\|^p d\gamma(x,y) \Bigg)^{1/p}
  \]
  \item Applications :
  \begin{itemize}
    \item Comparaison distributions réelles vs générées
    \item Génération de données réalistes avec structures géométriques
  \end{itemize}
\end{itemize}
\end{frame}

%================= Slide 30 : IA générative et OT =================
\begin{frame}{Applications à l’IA générative}
\begin{itemize}
  \item Score-based diffusion : génération par gradient de log-densité
  \item Transport optimal : mesurer distances entre distributions générées et réelles
  \item Topologie persistante : régulariser génération pour préserver cycles/structures
  \item Graphes et variétés : génération de molécules, maillages 3D, images structurées
\end{itemize}
\end{frame}

%================= Slide 31 : Synthèse =================
\begin{frame}{Module 4 : synthèse}
\begin{itemize}
  \item Topologie : Mapper, Persistent Homology → analyser la forme globale des données
  \item Géométrie avancée : variétés riemanniennes et distances géodésiques
  \item Transport optimal : distance de Wasserstein pour comparer distributions
  \item Applications IA générative : score-based diffusion, OT, régularisation topologique
\end{itemize}
\end{frame}

\end{document}


